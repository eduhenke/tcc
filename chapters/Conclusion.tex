\chapter{Conclusion}

In this thesis, we aimed to examine the potential advantages of utilizing advanced type systems, such as dependent types, in software development through the implementation of a dependently-typed programming language. Our specific objectives included investigating existing work on dependently-typed languages, developing a proof-of-concept language, using the language to encode business invariants and demonstrate the ability to prove properties about the code.

To support these objectives, we presented a dependently-typed programming language with a type-checker and a parser, written in Haskell. The language is based on the lambda calculus and has a simple syntax, with a type system that supports dependent types. While the language is not intended for production use, it serves as a proof-of-concept and a starting point for learning about dependently-typed programming languages.

Through the implementation and analysis of this language, we were able to gain a deeper understanding of how advanced type systems such as dependent types work and their potential benefits for improving the reliability and safety of software. However, it is important to note that the use of dependent types also comes with trade-offs and limitations, including the added burden of having to prove properties about the code and the potential for increased complexity in the codebase and development process. While the benefits of dependent types may outweigh the costs in certain cases, such as for critical or safety-sensitive systems, it is important to carefully evaluate the trade-offs and limitations for each specific application.

Overall, our findings suggest that dependent types have the potential to significantly improve the reliability and safety of software, but more research and development is needed to fully realize this potential in practice. The language and related resources can be found at \url{https://github.com/eduhenke/dep-tt}.
