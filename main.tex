% !TEX program = lualatex

\PassOptionsToPackage{dvipsnames}{xcolor}
\documentclass[
       embeddedlogo,
       english,
       lmodern,
       coorientadorbanca,
       noabntexcite
]{ufsc-thesis-rn46-2019}
       
\usepackage[dvipsnames]{xcolor}
\usepackage[OT1,LGR]{fontenc} % fontes
% \usepackage[utf8]{inputenc} % UTF-8
\usepackage{csquotes}
\usepackage{lipsum} % Gerador de texto
\usepackage{pdfpages} % Inclui PDF externo (ficha catalográfica)
\usepackage{syntax} % Usado para gramáticas

% \usepackage[section]{placeins} % So figures don't go to random places
\usepackage{float}

\usepackage{fontspec}
\usepackage{trfrac} % inference rules
\usepackage{amsthm} % add definitions
\theoremstyle{definition}
\newtheorem{definition}{Definition}[section]


\usepackage{amsmath}
\usepackage{mathtools}
\usepackage{textgreek}
% \usepackage[utf8]{inputenx}
% \usepackage[OT1,LGR]{fontenc}% changed the encoding


% Usado para mostrar código
\usepackage{listings}
\usepackage{lstpi} % pi-forall language
\usepackage{lsthaskell} % haskell language
% \usepackage{minted}
% \newmintinline[mt]{latex}{fontsize=\normalsize}
% \newmintinline[mft]{latex}{fontsize=\footnotesize}
% \setminted{fontsize=\small}
% \setmintedinline{breaklines,breakbytokenanywhere}

\newfontfamily\scpfamily{SourceCodePro}[
  NFSSFamily=sourcecodepro,
  Path=./fonts/sourcecodepro/,
  OpticalSize=18,
  Scale=MatchLowercase,
  UprightFont=*-Medium,
  ItalicFont=*-MediumIt,
  BoldFont=*-Bold,
  BoldItalicFont=*-BoldIt,
  FontFace = {el}{\shapedefault}{*-ExtraLight},
  FontFace = {el}{it}{*-ExtraLightIt},
  FontFace = {l}{\shapedefault}{*-Light},
  FontFace = {l}{it}{*-LightIt},
  FontFace = {sl}{\shapedefault}{*-Regular},
  FontFace = {sl}{it}{*-It},
  FontFace = {m}{\shapedefault}{*-Medium},
  FontFace = {m}{it}{*-MediumIt},
  FontFace = {sb}{\shapedefault}{*-Semibold},
  FontFace = {sb}{it}{*-SemiboldIt},
  FontFace = {b}{\shapedefault}{*-Bold},
  FontFace = {b}{it}{*-BoldIt},
  FontFace = {ub}{\shapedefault}{*-Black},
  FontFace = {ub}{it}{*-BlackIt},
]

\usepackage[style=abnt]{biblatex}
\addbibresource{bibliography.bib}
% \bibliography{bibliography.bib}

% Usado para mostrar comandos latex nesse guia:
\newcommand{\lacmd}[1]{\texttt{\textbackslash{}#1}}
\newcommand{\laenv}[1]{\texttt{\textbackslash{}begin\{#1\}...\textbackslash{}end\{#1\}}}
\newcommand{\laenvi}[2]{\texttt{\textbackslash{}begin\{#1\}[#2]...\textbackslash{}end\{#1\}}}

\newcommand{\code}[1]{\text{\scpfamily\setlength\spaceskip{0.35em}#1}}

\newcommand{\fnarrow}{\rightarrow}
\newcommand{\evalarrow}{\rightarrow}
\newcommand{\substarrow}{\mapsto}
\newcommand{\tylit}{\textbf{Type}}
\newcommand{\sep}{\begin{center}$\ast$~$\ast$~$\ast$\end{center}}
\newcommand{\infers}{\Rightarrow}
\newcommand{\checks}{\Leftarrow}

% changes math display font
\everymath{\mathtt{\xdef\tmp{\fam\the\fam\relax}\aftergroup\tmp}}
\everydisplay{\mathtt{\xdef\tmp{\fam\the\fam\relax}\aftergroup\tmp}}

%%%%%%%%%%%%%%%%%%%%%%%%%%%%%%%%%%%%%%%%%%%%%%%%%%%%%%%%%%%%%%%%%%%%
%%% Configurações da classe (dados do trabalho)                  %%%
%%%%%%%%%%%%%%%%%%%%%%%%%%%%%%%%%%%%%%%%%%%%%%%%%%%%%%%%%%%%%%%%%%%%

% Informações para capa e folha de rosto/certificacao

% Caso o título contenha alguma porção LaTeX ilegível, defina um título
% alternativo opcional com []'s para ser usado no campo Title do PDF
% IMPORTANTE: Os títulos deveriam ser iguais. Apenas use um título
% alternativo se o título não puder ser expresso com letras e números
\titulo[Implementing a programming language with a dependent type system]{Implementing a programming language with a dependent type system}

\autor{Eduardo Henke}
\data{\today}
\instituicao{Universidade Federal de Santa Catarina}
\centro{Centro Tecnológico}
\local{Florianópolis} % Apenas cidade! Sem estado
\programa{Programa de Pós-Graduação em Ciência da Computação}
% Os dois próximos itens são usados para gerar o \preambulo
% \tese % ou \dissertacao ou \tcc
% \titulode{graduação em Ciência da Computação}

%%% Atenção! No caso de TCC, além de usar \tcc, outros comandos devem ser fornecidos:
%%%
\tcc
\departamento{Departamento de Informática e Estatística}
\curso{Ciência da Computação}
\titulode{Bacharel em Ciência da Computação}

% Orientador, coorientador, membros da banca e coordenador
% As regras da BU agora exigem que Dr. apareça **depois** do nome
% Dica: para gerar Profᵃ. use Prof\textsuperscript{a}.
% Dica 2: para feminino use \orientadora e \coorientadora
\orientador{Prof. Alvaro Junio Pereira Franco, Dr.}
\coorientadorext{Li-Yao Xia, Dr.}{University of Edinburgh}
\membrabanca{Prof\textsuperscript{a}. Jerusa Marchi, Dr.}{Universidade Federal de Santa Catarina}
\membrobanca{Prof. Maicon Rafael Zatelli, Dr.}{Universidade Federal de Santa Catarina}
\membrobanca{Li-Yao Xia, Dr.}{University of Edinburgh}
% Dica: se feminino, \coordenadora
\coordenador{Prof. Jean Everson Martina, Dr}

\begin{document}
%%%%%%%%%%%%%%%%%%%%%%%%%%%%%%%%%%%%%%%%%%%%%%%%%%%%%%%%%%%%%%%%%%%%
%%% Principais elementos pré-textuais                            %%%
%%%%%%%%%%%%%%%%%%%%%%%%%%%%%%%%%%%%%%%%%%%%%%%%%%%%%%%%%%%%%%%%%%%%

% Inicia parte pré-textual do documento capa, folha de rosto, folha de
% aprovação, aprovação, resumo, lista de tabelas, lista de figuras, etc.
\pretextual%
\imprimircapa%
\imprimirfolhaderosto*
% Atenção! \cleardoublepages são inseridos automaticamente
% Atenção! esse \protect é importante
% \protect\incluirfichacatalografica{ficha-regenerate-later.pdf}
\imprimirfolhadecertificacao

\begin{resumo}[Resumo]
       O objetivo principal desse projeto é implementar uma linguagem de programação com um sistema de \emph{tipos dependentes}.
       Extende-se o cálculo \emph{lambda} com \emph{tipos dependentes}, o que possibilita a escrita de programas que não apenas realizam computações, mas também que permite a prova da corretude de seu comportamento.

       \vspace{\baselineskip}
       \textbf{Palavras-chave:} Tipos dependentes, sistema de tipos, cálculo lambda, linguagem de programação, teoria de tipos.
\end{resumo}

\begin{abstract}
       The main goal of this project is to design and implement a \emph{dependently typed} programming language.
       This work consists in an extension of the lambda calculus with dependent types, which allows us to write programs that not only have the ability to perform computations, but whose correctness can also be proven.

       \vspace{\baselineskip}
       \textbf{Keywords:} Dependent types, type system, lambda calculus, programming language, type theory.
\end{abstract}

% Listas de "coisas". O * no final faz com que as listas não sejam 
% incluídas como entratas do sumário (\tableofcontents)
% \listoftables*
% \listofalgorithms*
\listoffigures*
\tableofcontents*

%%%%%%%%%%%%%%%%%%%%%%%%%%%%%%%%%%%%%%%%%%%%%%%%%%%%%%%%%%%%%%%%%%%%
%%% Corpo do texto                                               %%%
%%%%%%%%%%%%%%%%%%%%%%%%%%%%%%%%%%%%%%%%%%%%%%%%%%%%%%%%%%%%%%%%%%%%
\textual%

\chapter{Introduction}

Software development is a big area, which it is being progressively valued over time.
However, due to the complexity of maintaining and developing large software systems, \textit{bugs} are frequent.
Where there is a discrepancy in what the developer expected to happen, with what was written in the software.

Because of this, we can see the importance of being able to guarantee that the software works as we expect, and it is also for this that the semantic analysis part of a language is used, where we can verify if the behavior expected by the developer reflects with what was in fact written.

In the semantic analysis phase, the compiler checks if the code written by the developer is in accordance with the problem modeling. One of the most basic mechanisms for this is \emph{types}, which we use to define which variables are of a given set of possible values.

With a simple type system~\cite{tapl}, we can verify that a variable of type \emph{string} cannot receive a value of type \emph{int}.

With a more advanced type system, i.e. with dependent types ~\cite{advancedtapl} we can:
\begin{itemize}
       \item specify business invariants that will be statically checked in the code, for example:
       \begin{itemize}
              \item in a banking system, during a withdrawal operation, the amount withdrawn cannot be greater than the balance from a bank account:
             \begin{piforall}
--  given the account balance, a amount and a proof
-- that the amount is less than the balance
-- perform the operation
withdraw_from :
       (account : Account) ->
       (amount : Nat) ->
       (amount <= account.balance) ->
       Nat
             \end{piforall}
             \item in most programs, we can have a list of elements, and we can have a function that returns the first element of the list, but what happens if the list is empty? We can use dependent types to specify that the list cannot be empty\footnote{In this case, the first two parameters are being passed explicitly, which can be cumbersome to the developer's experience, that's why most dependently-typed languages have some \emph{inference} mechanisms, which allows the compiler to infer some parameters being passed, like the element type and length of the vector}:
             \begin{piforall}

head :
       -- given any type A
       (A : Type) ->
       -- given any number
       (n : Nat) ->
       -- given a vector of type A, with length n+1
       -- (note that this means that even if n is 0,
       --- succ n, will be 1, and the vector will have
       -- at least one element, our language
       -- STATICALLY INVALIDATES  all uses of head on an empty list )
       Vec A (succ n) ->
       -- return the first element of the vector
       A

-- returns the length of the resulting vector as a type
append : (A : Type) -> (n : Nat) -> (m : Nat) -> Vec A m -> Vec A n -> Vec A (plus m n)
append = ... -- implementation is ommitted
             \end{piforall}
       \end{itemize}
       \item prove properties (and theorems) about the code (e.g. prove that an operation inserting an element into an ordered list does not change the order of the list).
\end{itemize}

\section{Existing work}
Famous examples of dependently-typed languages are \emph{Agda}~\cite{agda}, \emph{Idris}~\cite{idris}, \emph{Coq}~\cite{coq} and \emph{Lean}~\cite{lean}.

\subsection{Data types}
One example of a Agda program is shown below:
\begin{piforall}
data Nat : Set where
       zero : Nat
       suc : Nat -> Nat

_+_ : Nat -> Nat -> Nat
zero + m = m
suc n + m = suc (n + m)
\end{piforall}

We have defined a data type \code{Nat} that has two constructors, \code{zero} and \code{suc} of a \code{Nat}, which are used to represent the natural numbers.
Addition is defined as a function that takes two \code{Nat} and returns a \code{Nat}, by recursively removing the \code{suc} from the first parameter until it reaches \code{zero}.

We'll show how a vector and a function to get its first element can be defined in Agda:

\begin{piforall}
data Vec (A : Set) : Nat -> Set where
       [] : Vec A zero
       _::_ : {n : Nat} -> A -> Vec A n -> Vec A (suc n)

head : {A : Set}{n : Nat} -> Vec A (suc n) -> A
head (x :: xs) = x
\end{piforall}

We have defined \code{Vec}, which is a \emph{type constructor}, not a type by itself. We need to apply two arguments for it to be a type, the first one is a \emph{type} representing the vector's element's type and the second one is a \emph{natural number}, representing the vector's length. We now have that \code{Vec Nat 3} is a type, and that \code{Vec Nat 2} is also a type, but they are both different from each other.

We have also defined two ways to build \emph{instances} of the \code{Vec A n} type, the first one is the empty vector, which is represented by the \code{[]} constructor, it returns an instance of the type \code{Vec A zero} for any type \code{A}. And the second one is the \code{_::_} constructor, which takes an \emph{implicit parameter} \code{n} representing the length of the vector, a element of type \code{A}, a vector of type \code{Vec A n}, and returns a vector of type \code{Vec A (suc n)}. An example of a vector is \code{5 :: []}, which has type \code{Vec Nat (suc zero)}.

The \code{head} function takes an implicit parameter \code{A} representing the inner type of the vector, another implicit parameter \code{n} representing the length of the vector, a vector of type \code{Vec A (suc n)} and returns an element of type \code{A}. Because the vector is statically guaranteed to not be empty, Agda can check that the only constructor that can generate an element of type \code{Vec A (suc n)}, is the \code{_::_} constructor, with that we can use it to get the first element of the vector.

% TODO: explain and fix unicode
% \subsection{Proof as code}

% \begin{piforall}
% +-assoc : forall (m n p : Nat) -> (m + n) + p ≡ m + (n + p)
% +-assoc zero n p =
%   begin
%     (zero + n) + p
%   ≡⟨⟩
%     n + p
%   ≡⟨⟩
%     zero + (n + p)
%   ∎
% +-assoc (suc m) n p =
%   begin
%     (suc m + n) + p
%   ≡⟨⟩
%     suc (m + n) + p
%   ≡⟨⟩
%     suc ((m + n) + p)
%   ≡⟨ cong suc (+-assoc m n p) ⟩
%     suc (m + (n + p))
%   ≡⟨⟩
%     suc m + (n + p)
%   ∎

% \end{piforall}

\section{This work}


This work aims to design and implement a programming language with such dependent type system. It will provide a deeper understanding of how such languages work behind the scenes.
As the focus is on the type system, we will present only the type-checker of such language, that will be used to statically verify properties about a program.

\chapter{Theoretical basis}

To explain how this thesis is implemented we're going to provide some background in lambda calculus, inference rules and some type theories.

\section{Lambda Calculus}

When we don't have the tool of abstraction, calculations such as of Figure~\ref{fig:sample-calculation} seem complex to follow.
That's why we can abstract the underlying common concepts and define a function to capture that common abstraction(as defined in Figure~\ref{fig:factorial-def}).

\begin{figure}[H]
       \begin{equation*}
              \code{(3*2*1) + (7*6*5*4*3*2*1) + (5*4*3*2*1)}
       \end{equation*}
       \caption{Calculation of a number}
       \label{fig:sample-calculation}
\end{figure}


\begin{figure}[H]
       \begin{equation*}
              \code{factorial = $\lambda$n . if n == 0 then 0 else n * (factorial (n - 1))}
       \end{equation*}
       \caption{Definition of the factorial function}\label{fig:factorial-def}
\end{figure}

Now we can use the previously defined function to more concisely define the previously cumbersome calculation, as shown in Figure~\ref{fig:factorial-calc}.

\begin{figure}[H]
       \begin{equation*}
              \code{(factorial 3) + (factorial 7) + (factorial 5)}
       \end{equation*}
       \caption{Simplified calculation of a number}
       \label{fig:factorial-calc}
\end{figure}


We've captured an essential understanding of the calculation, and abstracted it into a concept that we can later reuse.

Lambda calculus captures the essential mechanisms of a programming language based on a few simple rules (abstraction, application). It was invented by Alonzo Church in the 1920s~\cite{tapl} and it is a \emph{logical calculus} or \emph{formal system}, whose terms are generated by the following grammar:

\begin{figure}[H]
       \begin{equation*}
              \begin{aligned}
                     t ::= & \ x            & \text{variable}    \\
                     |     & \  \lambda x.t & \text{abstraction} \\
                     |     & \  t\ t        & \text{application}
              \end{aligned}
       \end{equation*}
       \caption{Lambda calculus grammar}\label{fig:lambda-calc-grammar}
\end{figure}

A few example terms of the above grammar:

\begin{itemize}
       \item $x$, a simple(unbounded) variable
       \item $\lambda x.x$, the identity function
       \item $\lambda x.\lambda y.x$
       \item $(\lambda x.x)\ y$, applying $y$ to the identity function
\end{itemize}

\subsection{Evaluation rules}

This calculus primary benefit is in its evaluation, or computation.
For example, let's take the above mentioned term and evaluate it: $(\lambda x.x)\ y$ in one step evaluates to $y$, because we're applying the identity function to the variable $y$.
But how does that work for all terms? How can we formalize it?

\begin{definition}[Evaluation]
       $a \evalarrow b$, a term $a$ can evaluate to term $b$
\end{definition}

For that we need inference rules. Inference rules are syntactic transformation rules, i.e. they only need to check the form of the term and with that form they transform the original term to something else. We denote them like this, having the premises\footnote{when we don't need any pre-conditions we simply write nothing above the bar} above the bar and the conclusions/consequent below it:

\[\trfrac[(Add-Zero)]{a+0}{a}\]

\begin{definition}[Value]
       For a term to be a value, it must have the form of a function, i.e. only terms that follow the following structure are considered to be values: $\lambda x.t$. We denote them as terms beginning with \emph{v}, e.g. $v_1$
\end{definition}
\begin{definition}[Substitution]
       The process of substituting a variable $x$ for another term $y$ in a given term $t$ is denoted by $[x\substarrow y]t$
\end{definition}

\sep

We'll be denoting the evaluation of a lambda calculus terms using a system of inference rules written below:

\[
       \trfrac[(E-App1)]
       {t_1 \evalarrow t_1'}
       {t_1\ t_2 \evalarrow t_1'\ t_2}
\]

Given a term $t_1\ t_2$(applying the argument $t_2$ to the function $t_1$), if the term $t_1$ can evaluate to another term $t_1'$, we can rewrite the original term to $t_1'\ t_2$.


\[
       \trfrac[(E-App2)]
       {t_2 \evalarrow t_2'}
       {v_1\ t_2 \evalarrow v_1\ t_2'}
\]

Given a term $v_1\ t_2$(applying the argument $t_2$ to a term which is a value $v_1$), if the term $t_2$ can evaluate to another term $t_2'$, we can rewrite the original term to $v_1\ t_2'$.

\[
       \trfrac[(E-AppAbs)]
       {}
       {\lambda x.t_{12}\ v_2 \evalarrow [x \substarrow v_2]t_{12}}
\]

Whenever we have an application and the argument is a value, we can rewrite the original term to the body of the function, while replacing the parameter($x$) to the argument($v_2$).\footnote{This rule is also called a \emph{beta reduction}~\cite{tapl}.}\footnote{Note that we don't have any preconditions for this rule, whenever we \emph{syntactically} have an application whose argument is a value, we can reduce it.}

\begin{definition}[Normal form]
       When we have no more evaluation steps to perform, i.e. none of the above rules can be applied, and the resulting term is a \emph{value} we say that the term is in \emph{normal form}. For example the term $\lambda x.x$ is in normal form.
\end{definition}

At first it does not seem that this simple system can compute the same terms that an ordinary programming language can, but it turns out we can encode numbers, data structures(lists, sets, etc.), \emph{if} expressions~\cite{tapl}.
It was proved that this model is equivalent to a Turing Machine~\cite{lambda-church}.

\section{Simply-Typed Lambda Calculus}

Up until now, we've only had the notion of evaluation of terms, we're able to define computation steps and an engine can run those for us.
But without much insight into what \emph{"types"} of terms we're writing we can easily make the underlying evaluation engine try to compute a program that never halts~\cite{tapl}, or if we add to the lambda calculus' rules operations that can only be applied to numbers and feed them with booleans that can also make the engine stuck, i.e. passing a program that can't be computed.

Those problems can be solved if we somehow inspect the original program before evaluating it, and if desired properties can be derived from only the program specification, checking those properties against the code. One way to do it is by the concept of \emph{types}~\cite{tapl}.

Types allow us to define a set of possible values a term may have during runtime. When we talk about a variable having a type of \code{Nat}(Natural numbers set), it is telling that during runtime that variable can only possess the values \code{0, 1, 2, ...}. When we define a variable of type \code{Bool}, the only possible values in runtime are \code{True, False}.

We can also annotate various parts of our program that allows this checker(which we'll call typechecker from now), to verify the correctness of our program.

\subsection{Extension of the calculus}

Consider this extended version of lambda calculus representing the typed extension of the lambda calculus, we'll have a grammar for type construction(being represented by \code{T}, with some examples like \code{Nat} and \code{Nat \fnarrow Bool}), we'll annotate function parameters with their types \code{\lambda x:T. t}, and we'll have a typing context $\Gamma$ to record previous typing associations, as well as some terms of type \code{Nat} and \code{Bool}(\code{0}, \code{true}):

\begin{figure}[H]
       \[
              \begin{aligned}
                     t ::= & \ x                        \\
                     |     & \  \lambda x:T\ .\ t       \\
                     |     & \  t\ t                    \\
                     |     & \  true                    \\
                     |     & \  false                   \\
                     |     & \  if\ t\ then\ t\ else\ t \\
                     |     & \  0                       \\
                     |     & \  succ\ t
              \end{aligned}
              \begin{aligned}
                     T ::=      & \ Nat                \\
                     |          & \ Bool               \\
                     |          & \ T \fnarrow\ T      \\
                     \\
                     \Gamma ::= & \ \emptyset          \\
                     |          & \ \Gamma, x:T        \\
                     v ::=      & \  \lambda x:T\ .\ t \\
                     \\
              \end{aligned}
       \]
       \caption{Extended lambda calculus grammar}\label{fig:ext-lambda-calc-grammar}
\end{figure}

With its evaluation rules as stated in Figure~\ref{fig:ext-lambda-calc-eval-rules}:
\begin{itemize}
       \item Rule $E-IfTrue$, whenever we \emph{syntactically} find an if expression, whose condition is $true$, we can evaluate that expression to the first branch(\emph{then}) of the if expression, i.e. $t_2$.
       \item Rule $E-IfFalse$, whenever we \emph{syntactically} find an if expression, whose condition is $false$, we can evaluate that expression to the second branch(\emph{else}) of the if expression, i.e. $t_3$.
       \item Rule $E-Succ$, whenever we \emph{syntactically} find a $succ$ expression on a term $t$, we have to check the condition that $t$ evaluates to another term $t'$, if that's the case we can evaluate the $succ\ t$ expression to $succ\ t'$.
\end{itemize}

\begin{figure}[H]
       \[
              \begin{gathered}
                     \trfrac[(E-App1)]{t_1 \evalarrow t_1'}{t_1\ t_2 \evalarrow t_1'\ t_2} \\
                     \trfrac[(E-App2)]{t_2 \evalarrow t_2'}{v_1\ t_2 \evalarrow v_1\ t_2'} \\
                     \trfrac[(E-AppAbs)]{}{\lambda x.t_{12}\ v_2 \evalarrow [x \substarrow v_2]t_{12}} \\
                     \trfrac[(E-IfTrue)]{}{if\ true\ then\ t_2\ else\ t_3 \evalarrow t_2} \\
                     \trfrac[(E-IfFalse)]{}{if\ false\ then\ t_2\ else\ t_3 \evalarrow t_3} \\
                     \trfrac[(E-Succ)]{t \evalarrow t'}{succ\ t \evalarrow succ\ t'} \\
              \end{gathered}
       \]
       \caption{Extended lambda calculus evaluation rules}\label{fig:ext-lambda-calc-eval-rules}
\end{figure}

\subsection{Typing rules}

The typing relation for the \emph{extended} part of the extended lambda calculus, written $t : T$ is defined by inference rules assigning types to terms as stated in Figure~\ref{fig:ext-lambda-calc-typing-rules}:
\begin{itemize}
       \item Rule $T-True$, whenever we \emph{syntactically} find a $true$ term, we can assign it the type $Bool$.
       \item Rule $T-False$, whenever we \emph{syntactically} find a $false$ term, we can assign it the type $Bool$.
       \item Rule $T-If$, whenever we \emph{syntactically} find an if expression, we need to check the premises that the condition is a $Bool$ term, and the two branches are of the same type. If that's the case, we can assign the if expression the type of the two branches.
       \item Rule $T-Zero$, whenever we \emph{syntactically} find a $0$ term, we can assign it the type $Nat$.
       \item Rule $T-Succ$, whenever we \emph{syntactically} find a $succ$ expression on a term $t_1$, we have to check the condition that $t_1$ has type $Nat$, if that's the case we can assign the $succ\ t_1$ expression the type $Nat$.
\end{itemize}

\begin{figure}[H]
       \[
              \begin{gathered}
                     \trfrac[(T-True)]{}{true : Bool} \\
                     \trfrac[(T-False)]{}{false : Bool} \\
                     \trfrac[(T-If)]
                     {t_1 : Bool \qquad t_2 : T \qquad t_3 : T}
                     {if\ t_1\ then\ t_2\ else\ t_3 : T} \\
                     \trfrac[(T-Zero)]{}{0 : Nat} \\
                     \trfrac[(T-Succ)]{t_1 : Nat}{succ\ t_1 : Nat}
              \end{gathered}
       \]
       \caption{Extended lambda calculus typing rules}\label{fig:ext-lambda-calc-typing-rules}
\end{figure}

\begin{definition}[Well-typed term]
       A term that we can assign a type to is called a \emph{well-typed} term. For example, the term $if \ true\ then\ 0\ else\ succ\ 0$ is well-typed, because it has type $Nat$, by the following proof tree:

       \[
              \trfrac[(T-If)]
              {true : Bool \qquad 0 : Nat \qquad \trfrac[(T-Succ)]{0 : Nat}{succ\ 0 : Nat}}
              {if\ true\ then\ 0\ else\ succ\ 0 : Nat}
       \]
\end{definition}

Terms that are not well-typed, i.e. cannot be assigned a type given the rules above, for example the term $succ\ true$ won't evaluate to a value\footnote{we call those terms stuck, in a more formal definition, they are terms in normal form(no more evaluation rules apply) who are not  values}.

\sep

We've shown above how to check the types of the \emph{extended} part of the extended lambda calculus(e.g. for $true$, $0$, $succ$, etc.), but we haven't shown for the \emph{core} part of the lambda calculus(e.g. for $\lambda x.t$, $t_1\ t_2$, etc.).
For that we need the concept of a typing context, which will change the typing rules to include a context to aid in this process.

\begin{definition}{Context}
       A typing context $\Gamma$ is a sequence of variables and their associated types.
       The empty context is written as $\emptyset$.
       We can extend an existing context $\Gamma$ by adding a new variable with its associated type to it: $\Gamma, x : T$.\footnote{we will assume that each of the variables in this list are distinct from each other, so that there will always be at most one assumption about any variable's type.}

       The existing typing rule will change from a two-place relation $x : T$ to a three-place relation $\Gamma \vdash x : T$, meaning that $x$ has type $T$ under the context $\Gamma$, which will provide assumptions to the types of the free variables\footnote{variables that are not bound to any lambda binder} in $x$.
\end{definition}

The typing rules for the \emph{core} part of the lambda calculus, now written as $\Gamma \vdash t : T$ are defined in Figure~\ref{fig:core-lambda-calc-typing-rules}:
\begin{itemize}
       \item Rule $T-Var$, to check if a variable $x$ has type $T$ in the context $\Gamma$, we check if $x:T$ is in $\Gamma$.
       \item Rule $T-Abs$, a lambda expression $\lambda x:T_1.t_2$ has type $T_1 \fnarrow T_2$ in context $\Gamma$, if when we extend the context $\Gamma$ to include the variable $x$ with type $T_1$, we can check if $t_2$ has type $T_2$ in the extended context.
       \item Rule $T-App$, a function application $t_1\ t_2$ has type $T_{12}$ in context $\Gamma$, if we can check that $t_1$ has type $T_{11} \fnarrow T_{12}$ in the context $\Gamma$, and if we can check that $t_2$ has type $T_{11}$ in the context $\Gamma$.
\end{itemize}

\begin{figure}[H]
       \[
              \begin{gathered}
                     \trfrac[(T-Var)]{x:T \in \Gamma}{\Gamma \vdash x : T} \\
                     \trfrac[(T-Abs)]{\Gamma, x:T_1 \vdash t_2: T_2}{\Gamma \vdash \lambda x: T_1.\ t_2 : T_1 \fnarrow T_2} \\
                     \trfrac[(T-App)]{\Gamma \vdash t_1 : T_{11} \fnarrow T_{12} \qquad \Gamma \vdash t_2 : T_{11}}{\Gamma \vdash t_1\ t_2 : T_{12}}
              \end{gathered}
       \]
       \caption{Core lambda calculus typing rules}\label{fig:core-lambda-calc-typing-rules}
\end{figure}

With that we can now typecheck that the following program is valid:

\begin{figure}[H]
       $$ (\lambda x : Nat . succ\ x)\ (succ\ 0) $$
       \caption{Well-typed program}
\end{figure}
\begin{figure}[H]
       \[
              \trfrac[(T-App)]
              {
                     \trfrac[(T-Abs)]
                     {\Gamma, x:Nat \vdash \trfrac[(T-Succ)]{\trfrac[(T-Var)]{x:Nat \in \Gamma}{x : Nat}}{succ\ x : Nat}}
                     {\Gamma \vdash (\lambda x : Nat . succ\ x) : Nat \fnarrow Nat}
                     \qquad
                     \trfrac[(T-Succ)]{0 : Nat}{(succ\ 0) : Nat}
              }
              {(\lambda x : Nat . succ\ x)\ (succ\ 0) : Nat}
       \]\caption{Well-typed program typing proof tree}
\end{figure}

And we can't build a similar proof tree for the following program, because it is invalid:

\begin{figure}[H]
       $$ (\lambda x : Nat . succ\ x)\ true $$
       \caption{Not well-typed program}
\end{figure}

\section{Dependently-Typed Lambda Calculus}\label{dep-types}

The simply-typed system provides us some basic tools to define types, but we need to be able to define types for more complex terms.

Dependent types allow types to depend on the terms themselves, i.e. we don't have this stark distinction of types and terms. That allows us greater freedom in specifying types, which are a foundation on which our typechecker can verify the correctness of our code~\cite{advancedtapl}.

Like many dependently-typed languages, we'll show the typing rules(Figure~\ref{fig:dep-lambda-calc-typing-rules}) and a grammar(Figure~\ref{fig:dep-lambda-calc-grammar}) with the same \emph{syntax} for the terms and types, however for clarity we'll be using lowercase letters for terms and uppercase letters for their types.

\begin{figure}[H]
       \[
              \begin{aligned}
                     t, T ::= & \ x                  & \text{variable}                \\
                     |        & \  \lambda x.t       & \text{abstraction}             \\
                     |        & \  t\ t              & \text{application}             \\
                     |        & \ (t : T) \fnarrow T & \text{dependent function type} \\
                     |        & \ \tylit             & \text{the "type" of types}
              \end{aligned}
       \]
       \caption{Dependently-typed lambda calculus grammar}
       \label{fig:dep-lambda-calc-grammar}
\end{figure}

\begin{figure}[H]
       \[
              \begin{gathered}
                     \trfrac[(T-Var)]{x:T \in \Gamma}{\Gamma \vdash x : T} \\
                     \trfrac[(T-Lambda)]{\Gamma, x:T_1 \vdash y: T_2 \qquad \Gamma \vdash T_1 : \tylit}{\Gamma \vdash \lambda x.y : (x:T_1) \fnarrow T_2} \\
                     \trfrac[(T-App)]{\Gamma \vdash t_1 : (x:T_{11}) \fnarrow T_{12} \qquad \Gamma \vdash t_2 : T_{11}}{\Gamma \vdash t_1\ t_2 : [x \substarrow t_2]T_{12}} \\
                     \trfrac[(T-Pi)]{\Gamma \vdash T_1 : \tylit \qquad \Gamma, x: T_1 \vdash T_2 : \tylit}{\Gamma \vdash (x:T_1) \fnarrow T_2 : \tylit} \\
                     \trfrac[(T-Type)]{}{\Gamma \vdash \tylit : \tylit}
              \end{gathered}
       \]
       \caption{Dependently-typed lambda calculus typing rules}
       \label{fig:dep-lambda-calc-typing-rules}
\end{figure}

The typing rules $T-Var$, $T-Lambda$ and $T-App$ are very similar to those in the simply-typed lambda calculus, but the main difference is that the function type, has now a binder variable for the argument.
Where before we only had that the type of a function is $T_1 \fnarrow T_2$, now we have that the type of a function\footnote{formally called a \emph{Pi} type} is $(x:T_1) \fnarrow T_2$. That means that the function's return type expression $T_2$ can depend on the function's argument $x$, e.g. the type of a function that inserts a $Bool$ in a length-indexed vector of $Bool$s can be $(n:Nat) \fnarrow Bool \fnarrow Vec\ Bool\ (n+1)$.
The other rules are explained as follows:
\begin{itemize}
       \item Rule $T-Pi$, a Pi type $(x:T_1)\fnarrow T_2$ has type $\tylit$ in context $\Gamma$, only if $T_1$ also has type $\tylit$ in context $\Gamma$, and if $T_2$ has type $\tylit$ in the extended context $\Gamma, x:T_1$.
       \item Rule $T-Type$, the term $\tylit$ also has type $\tylit$.
\end{itemize}

We'll add two extensions to our grammar to aid in creating programs:

\begin{figure}[H]
       \[
              \begin{aligned}
                     t, T ::= ...                                         \\
                     | & \ t : T    & \ \text{type annotation}            \\
                     | & \ name = t & \ \text{assigning a name to a term}
              \end{aligned}
       \]
       \caption{Dependently-typed lambda calculus syntax sugar}
\end{figure}

Which respectively mean:
\begin{itemize}
       \item An expression can be annotated with a type, e.g. $x : Nat$, and that will trigger the typechecker to check that $\Gamma \vdash x : Nat$ given the underlying context.
       \item A name can be assigned to a term, e.g. $id = \lambda x.x$.
\end{itemize}

\sep

With that we can write this polymorphic identity function program annotated with its type and its associated typing proof tree:

$$
       \begin{aligned}
              id & : (x:\tylit) \fnarrow (y:x) \fnarrow x \\
              id & = \lambda x.\lambda y.y
       \end{aligned}
$$

We can derive a proof tree proving that the type of $id$ is in fact what was annotated:

$$
       \trfrac[(T-Lambda)]
       {
              \Gamma, x:\tylit \vdash
              \trfrac[(T-Lambda)]
              {
                     \Gamma, y:x \vdash y:x
                     \qquad
                     \trfrac[(T-Var)]{x:\tylit \in \Gamma}{x : \tylit}
              }
              {\lambda y.y : (y:x) \fnarrow x}
              \qquad
              \trfrac[(T-Type)]{}{\Gamma \vdash \tylit : \tylit}
       }
       {\Gamma \vdash \lambda x.\lambda y.y : (x:\tylit) \fnarrow (y:x) \fnarrow x}
$$

The goal of this project is to derive the typing proof tree automatically.

% TODO: explain proofs as types

% Using the existing grammar we can describe more advanced programs, that can be used to prove arbitrary mathematical/logical properties like the commutativity of the $and$ operator:

% $$
%        \begin{aligned}
%               and\            & :\ Type\ \fnarrow\ Type\ \fnarrow\ Type                                                     \\
%               and\            & =\ \lambda p.\ \lambda q.\ (c:\ Type)\ \fnarrow\ (p\ \fnarrow\ q\ \fnarrow\ c)\ \fnarrow\ c \\
%               \\
%               conj\           & :\ (p:\ Type)\ \fnarrow\ (q:Type)\ \fnarrow\ p\ \fnarrow\ q\ \fnarrow\ and\ p\ q            \\
%               conj\           & =\ \lambda p.\lambda q.\ \lambda x.\lambda y.\ \lambda c.\ \lambda f.\ f\ x\ y              \\
%               \\
%               proj1\          & :\ (p:\ Type)\ \fnarrow\ (q:Type)\ \fnarrow\ and\ p\ q\ \fnarrow\ p                         \\
%               proj1\          & =\ \lambda p.\ \lambda q.\ \lambda a.\ a\ p\ (\lambda x.\lambda y.x)                        \\
%               \\
%               proj2\          & :\ (p:\ Type)\ \fnarrow\ (q:Type)\ \fnarrow\ and\ p\ q\ \fnarrow\ q                         \\
%               proj2\          & =\ \lambda p.\ \lambda q.\ \lambda a.\ a\ q\ (\lambda x.\lambda y.y)                        \\
%               \\
%               and\_commutes\  & :\ (p:Type)\ \fnarrow\ (q:Type)\ \fnarrow\ and\ p\ q\ \fnarrow\ and\ q\ p                   \\
%               and\_commutes\  & =\ \lambda p.\ \lambda q.\ \lambda a.\ conj\ q\ p\ (proj2\ p\ q\ a)\ (proj1\ p\ q\ a)
%        \end{aligned}
% $$

% TODO: maybe finish this proof tree?
% $$
%        \trfrac[(T-Lambda)]
%        {
%               \Gamma, p:\tylit \vdash \lambda q.\ \lambda a.\ conj\ q\ p\ (proj2\ p\ q\ a)\ (proj1\ p\ q\ a) : (q:Type)\ \fnarrow\ and\ p\ q\ \fnarrow\ and\ q\ p
%               \qquad
%               \trfrac[(T-Type)]{}{\Gamma \vdash \tylit : \tylit}
%        }
%        {\lambda p.\ \lambda q.\ \lambda a.\ conj\ q\ p\ (proj2\ p\ q\ a)\ (proj1\ p\ q\ a) : (p:Type)\ \fnarrow\ (q:Type)\ \fnarrow\ and\ p\ q\ \fnarrow\ and\ q\ p}
% $$

\subsection{Definitional Type Equality}

\subsubsection{Motivation}

In languages with dependent types, it is often necessary to equate types that are not merely alpha-equivalent\footnote{alpha-equivalence is the property of two terms being equal are equivalent for all purposes if their only difference is the renaming of bound variables, e.g. $\lambda x.x$ is alpha-equivalent to $\lambda y.y$~\cite{nlab:alpha-equivalence}}. This is because more expressions need to type check in these languages.
For example, a type of length-indexed vector might be \code{Vec A n}, where \code{A} is the type of the vector's elements, and \code{n} is the length of the vector.
We could have a safe head operation that would allow us to access the first element of the vector, as long as it isn't empty and a append operation that would allow us to append elements to the vector.

\begin{piforall}
       head : (A : Type) -> (n : Nat) -> Vec A (succ n) -> A
       head = ... -- implementation is ommitted

       -- returns the length of the resulting vector as a type
       append : (A : Type) -> (n : Nat) -> (m : Nat) -> Vec A m -> Vec A n -> Vec A (plus m n)
       append = ... -- implementation is ommitted
\end{piforall}

Observe that the following program would compile with the existing theory:

\begin{piforall}
       v' : Vec Bool (succ 0)
       v' = Cons True VNil

       h : Bool
       h = head Bool 0 v'
\end{piforall}


Because the type of v' is \code{Vec Bool (succ 0)} matches exactly what the $head$ function expected:

\begin{piforall}
       head : (A : Type) -> (n : Nat) -> Vec A (succ n) -> A
       head Bool : (n : Nat) -> Vec Bool (succ n) -> Bool
       head Bool 0 : Vec Bool (succ 0) -> Bool
       head Bool 0 v' : Bool
\end{piforall}


However the application of \code{head Bool 1 (append Bool 1 1 v' v')}, wouldn't typecheck, observe why:

\begin{piforall}
       v' : Vec Bool (succ 0)
       append : (A : Type) -> (n : Nat) -> (m : Nat) -> Vec A m -> Vec A n -> Vec A (plus m n)

       append Bool 1 1 : Vec Bool 1 -> Vec Bool 1 -> Vec Bool (plus 1 1)
       append Bool 1 1 v' v' : Vec Bool (plus 1 1)

       head : (A : Type) -> (n : Nat) -> Vec A (succ n) -> A
       head Bool : (n : Nat) -> Vec Bool (succ n) -> Bool
       head Bool 1 : Vec Bool (succ 1) -> Bool

       -- would not type check because head expects:
       --   (Vec Bool (succ 1))
       -- but we have:
       --   (Vec Bool (plus 1 1))
       head Bool 1 (append Bool 1 1 v' v') : ???
\end{piforall}

It seems alpha-equivalence is not enough to type check this program, we need to be able to equate \code{Vec Bool (succ 1)} and \code{Vec Bool (plus 1 1)}. And that seems to require some number of steps of computation to be able to do so.
Definitional type equality is the tool we need to also equate those types of terms.

\subsubsection{Definition}

\begin{definition}[Definitional equality]
       Definitional equality is a judgement of the form: $\Gamma \vdash A = B$.
       Classically, definitional equality is called intensional equality\footnote{intensional equality is the relation generated by abbreviatory definitions, changes of bound variables and the principle of substituting equals for equals~\cite{nlab:equality}}. However in this paper we'll define it to mean both intensional \textbf{and} computational equality\footnote{computational equality is the relation generated by various reduction rules, e.g. \emph{beta reduction}~\cite{nlab:equality}}~\cite{nlab:equality}.
\end{definition}

This judgement is defined by the properties stated in Figure~\ref{fig:def-eq}. Rule $E-Beta$ ensures that beta-equivalence is contained in this judgement, because terms that evaluate to each other should be equal.
Rules $E-Refl$, $E-Sym$, and $E-Trans$ allows this judgement to be considered an equivalence relation~\cite{oplss}.


\begin{figure}[H]
       $$
              \begin{gathered}
                     \trfrac[(E-Beta)]{}{\Gamma \vdash (\lambda x.a)\ b = [x \substarrow b]a} \\
                     \trfrac[(E-Refl)]{}{\Gamma \vdash A = A} \\
                     \trfrac[(E-Sym)]{\Gamma \vdash A = B}{\Gamma \vdash B = A} \\
                     \trfrac[(E-Trans)]
                     {\Gamma \vdash A_1 = A_2 \qquad \Gamma \vdash A_2 = A_3}
                     {\Gamma \vdash A_1 = A_3} \\
                     \trfrac[(E-Pi)]
                     {\Gamma \vdash A_1 = A_2 \qquad \Gamma,x:A_1 \vdash B_1 = B_2}
                     {\Gamma \vdash (x:A_1) \fnarrow B_1 : (x:A_2) \fnarrow B_2} \\
                     \trfrac[(E-Lam)]
                     {\Gamma,x:A_1 \vdash b_1 = b_2}
                     {\Gamma \vdash \lambda x.b_1 : \lambda x.b_2} \\
                     \trfrac[(E-App)]
                     {\Gamma \vdash a_1 = a_2 \qquad \Gamma \vdash b_1 = b_2}
                     {\Gamma \vdash a_1\ b_1 = a_2\ b_2} \\
                     \trfrac[(E-Lift)]
                     {\Gamma, x:A \vdash b : B \qquad \Gamma \vdash a_1 = a_2}
                     {\Gamma \vdash [x \substarrow a_1]b = [x \substarrow a_2]b} \\
                     \trfrac[(E-Annot)]
                     {\Gamma \vdash a_1 = a_2}
                     {\Gamma \vdash (a_1: A) = a_2}
              \end{gathered}
       $$
       \caption{Inference rules for the definitional type equality judgement}
       \label{fig:def-eq}
\end{figure}

With those rules we can now successfully type-check the program, because $plus\ 1\ 1$ will evaluate to $succ\ 1$.

% \subsection{Propositional Type Equality}
% \subsubsection{Motivation}
% \subsubsection{Definition}

\chapter{Implementation}

The rules presented on section \ref{dep-types} were developed to allow typing proof trees to be built, however they were not developed having in mind \emph{how} these proof trees can be built, i.e. these rules are not syntax-directed, we can not devise a decidable algorithm based on these rules alone.
The reason for that is because rule $T-Lambda$ is not syntax-directed, it is not clear what is the type of the argument of the function(we extend the context with it, because it is necessary to typecheck the body of the function).
Because of that we need to revise the existing rules and transform them into syntax-directed rules.

\section{Bidirectional type system}

\begin{definition}[Judgement]
       A judgement is a proposition that is made on a given term. The previously defined \emph{typing rule} is a form of judgement.\cite{nlab:judgment}
\end{definition}

One way to do that is to define a bidirectional type system, which is based on two types of judgements:
\begin{itemize}
       \item \emph{Type inference}, denoted as $\Gamma \vdash x \infers T$ which given a term $x$ and a context $\Gamma$, will \emph{infer}(return) the type of that term.
             \begin{haskell}
                    inferType :: Context -> Term -> Maybe Type
             \end{haskell}
       \item \emph{Type checking}, denoted as $\Gamma \vdash x \checks T$ which given a term $x$, a context $\Gamma$ and a type $T$, will \emph{check} that the term is of the given type.
             \begin{haskell}
                    checkType :: Context -> Term -> Type -> Bool
             \end{haskell}
\end{itemize}

We now can develop inference rules that represent the algorithmically-feasible version of the \ref{dep-types} rules. Both types of judgements will be used, the type inference rules can use the type checking rules, and vice versa.

\begin{figure}[H]
       $$
              \begin{gathered}
                     \trfrac[(I-Var)]
                     {x:A \in \Gamma}
                     {\Gamma \vdash x \infers A} \\
                     \trfrac[(I-App-Simple)]
                     {
                            \Gamma \vdash a \infers (x:A) \fnarrow B
                            \qquad
                            \Gamma \vdash b \checks A
                     }
                     {\Gamma \vdash a\ b \infers [x \substarrow b]B} \\
                     \trfrac[(I-Pi)]
                     {
                            \Gamma \vdash A \checks \tylit
                            \qquad
                            \Gamma, x:A \vdash B \checks \tylit
                     }
                     {\Gamma \vdash (x:A) \fnarrow B \infers \tylit} \\
                     \trfrac[(I-Type)]{}{\Gamma \vdash \tylit \infers \tylit} \\
                     \trfrac[(I-Annot)]
                     {\Gamma \vdash a \checks A}
                     {\Gamma \vdash (a : A) \infers A}
              \end{gathered}
       $$
       \caption{Bidirectional type-inference rules}
       \label{fig:dep-lambda-type-inference}
\end{figure}

\begin{figure}[H]
       $$
              \begin{gathered}
                     \trfrac[(C-Lambda)]
                     {
                            \Gamma, x:A \vdash a \checks B
                            \qquad
                            \Gamma \vdash A \checks \tylit
                     }
                     {\Gamma \vdash \lambda x.a \checks (x:A) \fnarrow B} \\
                     \trfrac[(C-Infer-Simple)]
                     {\Gamma \vdash a \infers A}
                     {\Gamma \vdash a \checks A}
              \end{gathered}
       $$
       \caption{Bidirectional type-checking rules}
       \label{fig:dep-lambda-type-checking}
\end{figure}

\section{Modules}

We'll also introduce the concept of modules and top-level declarations to aid in the development of programs. Where a module consists of a list of declarations that are defined once and used throughout the module:

\begin{figure}[H]
       \[
              \begin{aligned}
                     Decl ::= & \ t : T & \text{type-signature} \\
                     |        & \ t = t & \text{definition}
              \end{aligned}
       \]
       \caption{Module grammar}
\end{figure}


Here's some examples:

\begin{piforall}
-- a type signature, linking "id" with that type
id : (a : Type) -> a -> a

-- a definition, linking "id" with that term/value
id = λa. λx. x
\end{piforall}

\subsection{Type-checking}

To type-check a module we use the \code{checkType} function when we have an associated type-signature for a given name, and we use the \code{inferType} function when we do not have a type-signature for a given name. Also we put the existing declarations in the scope of all subsequent type-checks.

\section{Data Types}

We'll also introduce the concept of data types, to aid the programmer in having structured data in their program:

\begin{figure}[H]
       \[
              \begin{aligned}
                     Decl ::= & \ ... \\
                     |        & \ \emph{data}\ TConName\ Telescope\ \emph{:\ Type\ where}\ ConstructorDef+ & \ \text{data definition}
              \end{aligned}
       \]
       \[
              \begin{aligned}
                     ConstructorDef ::= & \  DConName\ Telescope \\
                     TConName       ::= & \ identifier & \text{type constructor name} \\
                     DConName       ::= & \ identifier & \text{data constructor name} \\
                     Telescope      ::= & \ Decl*
              \end{aligned}
       \]
       \[
              \begin{aligned}
                     t, T ::= & \ ... \\
                     |        & \ TConName\ t* & \text{constructing a data type by applying a list of arguments}     \\
                     |        & \ DConName\ t* & \text{instance of a data type}
              \end{aligned}
       \]
       \caption{Data type grammar}
\end{figure}


\begin{definition}[Telescope]
       A telescope is a list of declarations, it is called like that because of the scoping behavior of this structure~\cite{oplss}.
       The scope of each variable merges with all of the subsequent ones, e.g.:
       \begin{piforall}
              -- notice the scoping behavior of this structure
              (A : Type) (n : Nat) (v : Vec A n)
              -- here we're showing that the telescope can use both
              -- a type annotation, and a definition, which symbolizes
              -- a constraint being put upon any variable, in this
              -- case, it requires n to be of value Zero
              (n : Nat) [n = Zero]
       \end{piforall}
\end{definition}

Here we're showing an example, by first defining some data types, then using it as terms:

\begin{piforall}
data Bool : Type where {
       False,  -- False is a case from Bool, it is a data type constructor
       True    -- as well as True
} -- Bool is a data definition with an empty telescope

data Nat : Type where {
       Zero,
       -- Succ is a data type constructor with
       -- a Telescope of one argument(another Nat number)
       Succ of (Nat)
}

data List (A : Type) : Type where {
       Nil,
       -- Cons is a data type constructor with
       -- a Telescope of two arguments(element of A and a List of A)
       Cons of (A) (List A)
}

-- t is annotated with a type, constructed by applying a
-- list of arguments(Bool) to the type constructor(List)
t : List Bool
-- t is defined using a data type constructor(Cons) by applying
-- a list of arguments(True, Nil) to it
t = Cons True Nil
\end{piforall}

Next, we're showing that the telescope can have a constraint on the previous values, with a \emph{Vector} data type:

\begin{piforall}
data Vec (A : Type) (n : Nat) : Type where {
  -- Nil is a data type constructor with
  -- a constraint that n(provided by the type) must be zero
  Nil of [n = Zero],
  -- Cons is a data type constructor with a Telescope of 
  -- three arguments(Nat m, element of A, and a Vec of A with length m)
  -- and one constraint that the Vector built by Cons,
  -- must have length m+1, m being the underlying vector
  -- which this Cons was built upon
  Cons of (m : Nat) (A) (Vec A m) [n = Succ m]
}

-- t is annotated with a type, constructed by applying a
-- list of arguments(Bool) to the type constructor(List)
t : Vec Bool 2
-- t is defined using a data type constructor(Cons) by applying
-- a list of arguments(True, Nil) to it
t = Cons 1 True (Cons 0 False Nil)
\end{piforall}

\subsubsection{Type-checking}

We type-check type and data-type constructor applications according to the telescope and the arguments provided, e.g.:

\begin{piforall}
-- compares the telescope (A : Type) (n : Nat)
-- with the arguments provided: Bool, 1
-- Bool is of type Type, and 1 is of type Nat
v : Vec Bool 1
-- compares the telescope (m : Nat) (A : Type) (v : Vec A m)
-- with the arguments provided: 0, True, Nil
-- 0 is of type Nat, True is of type A(Bool), and Nil is of type Vec Bool 0
v = Cons 0 True Nil
\end{piforall}

\subsection{Pattern-matching}

\begin{figure}[H]
       \[
              \begin{aligned}
                     t, T ::= & \ ... \\
                     |        & \ \emph{case}\ t\ \emph{of}\ Case* & \text{pattern matching of a term} \\
                     \\
                     Case      ::= & \ Pattern\ \rightarrow\ t \\
                     Pattern   ::= & \ DConName\ PatVar* \\
                     |             & \ PatVar   \\
                     PatVar    ::= & \ identifier
              \end{aligned}
       \]
       \caption{Data type with pattern matching grammar}
\end{figure}

Here's some examples of pattern matching:

\begin{piforall}
not : Bool -> Bool
not = λb. case b of {
  False -> True,
  True -> False
}

plus : Nat -> Nat -> Nat
plus = λa. λb. case a of {
  Zero -> b,
  Succ a' -> Succ (plus a' b)
}
\end{piforall}

\subsubsection{Type-checking}

To type-check pattern matching cases, we have to:

\begin{itemize}
       \item extend the typing context with the pattern variables, e.g. when we see \code{Succ n' -> body}, we should extend the typing context with \code{n' : Nat} when we type-check the body:
\begin{piforall}
n : Nat
case n of {
  -- we should know that n' has type Nat,
  --  according to Nat's data definition
  Succ n' -> plus n' n',
  ...
}
\end{piforall}
       \item all cases must conform to the same type, e.g.:
\begin{piforall}
nat_to_bool : Nat -> Bool
nat_to_bool = λn. case n of {
  -- the type of the body of this case is Bool,
  -- so the type of the whole case is Bool
  Zero -> False,
  -- the same thing with this case
  Succ n' -> True
}

-- invalid code would have been:
-- case n of { Zero -> False, Succ n' -> n' }
\end{piforall}
       \item check for exhaustiveness, i.e. we should check that there's no other possible case, otherwise, we should throw an error
\begin{piforall}
n : Nat
-- this case is exhaustive, because it covers all possible cases
case n of {
  Zero -> False,
  Succ n' -> True
}

-- this case is not exhaustive, because it doesn't cover the case of Zero
-- case n of {
--  Succ n' -> True
-- }
\end{piforall}
       \item unify the scrutinee's type(term being matched) with the type of the pattern, e.g. if we're applying a case expression to a \code{n} which is a \code{Nat}, and in the \code{Zero -> body} case, we should replace the type of \code{a} from \code{Nat} to \code{Zero}, in this branch.
\end{itemize}



\section{Project's Code}

The project was developed in \emph{Haskell} using the \emph{Unbound} library for variable substitution(\emph{beta reduction}).
\subsection{Terms}

The syntax of the terms, module and declarations are represented as a data type in \emph{Haskell}:

\begin{lstlisting}[language=Haskell]
module Syntax where

type TName = Unbound.Name Term -- Term names for our AST

type TCName = String -- type constructor names

type DCName = String -- data constructor names

-- because types and terms are the same in dependent typing,
-- we'll alias them
type Type = Term

data Term
       = Type -- type of types
       | Var TName -- variables: x
       | Lam (Unbound.Bind TName Term) -- abstractions: λx.a
       | App Term Term -- application: f x
       | Pi Type (Unbound.Bind TName Type) -- function types: (x : A) -> B
       | Ann Term Type -- "ascription" or "annotated terms": (a: A)
       -- Data-type related terms
       | DCon DCName [Term] -- Just True
       | TCon TCName [Term] -- Maybe Bool
       | Case Term [Match] -- case analysis  `case a of matches`
       -- Proof related terms
       | TyEq Type Type -- equality type: (plus 0 n) = n
       | Refl -- equality evidence: refl is of type x = x
       | Subst Term Term
       deriving (Generic)

-- Modules
newtype Module = Module {declarations :: [Decl]}
       deriving (Show)

-- a "top-level definition" of a module
data Decl
       = TypeSig TName Type -- a : A
       | Def TName Term -- a = b
       | DataDef TCName Telescope [ConstructorDef] -- data Bool ...
       deriving (Generic, Unbound.Alpha, Unbound.Subst Term)

-- Data-types

-- a data constructor has a name and a telescope of arguments
data ConstructorDef = ConstructorDef DCName Telescope
       deriving (Show, Generic, Unbound.Alpha, Unbound.Subst Term)

newtype Telescope = Telescope [Decl]
       deriving (Show, Generic, Unbound.Alpha, Unbound.Subst Term)

-- represents a case alternative
newtype Match = Match (Unbound.Bind Pattern Term)
       deriving (Generic)
       deriving anyclass (Unbound.Alpha, Unbound.Subst Term)

data Pattern
       = PatCon DCName [Pattern]
       | PatVar TName
       deriving (Eq, Generic, Unbound.Alpha, Unbound.Subst Term)
\end{lstlisting}

\subsection{Type-checking}

\subsubsection{Type-checking monad}
Instead of using a \code{Maybe} or \code{Either} type to represent the result of type checking, we'll use a monad to encapsulate the behaviour of:
\begin{itemize}
       \item failing to type-check and returning an error message
       \item getting and setting the state of the typing context
       \item generating fresh variable names
\end{itemize}

\begin{lstlisting}[language=Haskell]
import Control.Monad
import Control.Monad.Except (ExceptT)
import Control.Monad.Reader (ReaderT)
import qualified Unbound.Generics.LocallyNameless as Unbound

type TcMonad = Unbound.FreshMT (ReaderT Env (ExceptT Err IO))

data Env = Env {ctx :: [Decl]}

emptyEnv :: Env
emptyEnv = Env {ctx = []}

extendCtx :: Decl -> TcMonad a -> TcMonad a
extendCtxs :: [Decl] -> TcMonad a -> TcMonad a
lookupTyMaybe :: TName -> TcMonad (Maybe Type)
lookupTy :: TName -> TcMonad Type
-- implementation ...
\end{lstlisting}

We now have some helper functions that can be used to implement the type-checking rules, e.g. given a name of a variable we can use \code{lookupTy} to get its type, if it fails to find that variable in the context, it will stop the type-checking process and issue an error.

\subsubsection{Equality and weak-head normal form}

The rules for propositional and definitional equality of terms are defined in the \code{equate} function, which given two terms will check if they are equal or not, if they are not it will throw an error.

\begin{lstlisting}[language=Haskell]
equate :: Term -> Term -> TcMonad ()
-- two terms are equal, if they are alpha-equivalent
-- i.e., by just properly renaming the variables
-- they are the same term
equate t1 t2 | aeq t1 t2 = return ()
equate t1 t2 = do
  nf1 <- whnf t1
  nf2 <- whnf t2
  case (nf1, nf2) of
    (Lam bnd1, Lam bnd2) -> do
      -- get the body of each lambda
      (_, t1, _, t2) <- unbind2Plus bnd1 bnd2
      -- lambdas are equal, if their bodies are equal
      equate t1 t2
    -- application is equal when:
    -- - both functions are equal and
    -- - both args are equal
    (App f1 x1, App f2 x2) -> do
      equate f1 f2
      equate x1 x2
    -- ... rest of the terms ommited for brevity ...
    (_, _) -> err ["Expected", show nf2, "but found", show nf1]
\end{lstlisting}


\begin{definition}[Weak-head normal form]
       A term is said to be in weak-head normal form, when we apply weak-head normalization, which is when the leftmost, outermost reducible expression is always selected for beta-reduction, and the process is halted as soon as the term begins with something other than a lambda abstraction.~\cite{advancedtapl}

       An unformal idea of weak-head normalization, is that it is a subset of the normal beta-reduction(computation) rules, so that we compute just enough to observe the structure of the term, e.g. when I don't care about the result of \code{factorial 100}, but I do care about its structure, if it is a \code{Nat} or not.
\end{definition}

As shown in the above code section, we also need a function to calculate the weak-head normal form of a given term, which is defined as follows in the \code{whnf} function:

\begin{lstlisting}[language=Haskell]
whnf :: Term -> TcMonad Term
-- if we're whnf-inf a variable,
whnf (Var x) = do
  -- look up its definition in the context
  maybeTm <- lookupDefMaybe x
  case maybeTm of
    -- and whnf the definition
    (Just tm) -> whnf tm
    -- or if there's no definition in the context
    -- return the variable 
    _ -> pure (Var x)
-- if we're whnf-ing an application,
whnf (App a b) = do
  v <- whnf a
  -- check the type of the whnf'd function
  case v of
    -- if it is indeed a lambda abstraction
    (Lam bnd) -> do
      (x, a') <- unbind bnd
      -- substitute the argument for the bound variable
      whnf (subst x b a')
    -- otherwise, return the application
    _ -> return (App v b)
-- ... rest of the terms ommited for brevity ...
whnf tm = return tm
\end{lstlisting}

\subsubsection{Type-checking rules}
Also the two judgements' signatures will change to:

\begin{lstlisting}[language=Haskell]
-- given a term return its type(or an error message)
inferType :: Term -> TcMonad Type
-- given a term and a type, check if the term is of the given type
-- if it is, return (), otherwise an error message
checkType :: Term -> Type -> TcMonad ()
\end{lstlisting}

We'll use a \code{tcTerm} function that will be used to centralize the two judgements. If the second parameter is \code{Nothing}, it will enter into \emph{inference} mode, otherwise it will enter into \emph{type-checking} mode.

\begin{lstlisting}[language=Haskell]
inferType :: Term -> TcMonad Type
inferType t = tcTerm t Nothing

checkType :: Term -> Type -> TcMonad ()
checkType tm ty = do
  -- Whenever we call checkType we should call it
  -- with a term that has already been reduced to 
  -- normal form. This will allow rule c-lam to
  -- match against a literal function type.
  nf <- whnf ty
  ty' <- tcTerm tm (Just nf)

-- | Make sure that the term is a type (i.e. has type 'Type')
tcType :: Term -> TcMonad ()
tcType tm = void (checkType tm Type)

tcTerm :: Term -> Maybe Type -> TcMonad Type
-- Infer-mode
-- looks up the var type from context
tcTerm (Var x) Nothing = lookupTy x
-- when we reach an annotation, checks the type of the term
-- to be the annotated type
tcTerm (Ann tm ty) Nothing = do
  checkType tm ty
  return ty
tcTerm (Pi tyA bnd) Nothing = do
  (x, tyB) <- unbind bnd
  -- check that the argument type is indeed a type
  tcType tyA
  -- extend the context with x:A, then check
  -- that the body's return type is indeed a type
  extendCtx (TypeSig x tyA) (tcType tyB)
  return Type
tcTerm (App t1 t2) Nothing = do
  ty1 <- inferType t1
  let ensurePi :: Type -> TcMonad (TName, Type, Type)
      ensurePi (Ann a _) = ensurePi a
      ensurePi (Pi tyA bnd) = do
        (x, tyB) <- unbind bnd
        return (x, tyA, tyB)
      ensurePi ty = err ["Expected a function type, but found ", show ty]
  nf1 <- whnf ty1
  -- checks if the "function term" is indeed a function and
  -- gets the bounded arg var, the argument and body's type
  (x, tyA, tyB) <- ensurePi nf1
  -- checks that the provided argument t2, has the same type as the
  -- function's argument type
  checkType t2 tyA
  -- returns the function's body type, with the argument substituted
  return (subst x t2 tyB)
-- ... rest of the terms ommited for brevity ...
tcTerm tm Nothing = err ["Must have a type annotation to check ", show tm]

-- Check-mode
tcTerm (Lam bnd) (Just ty@(Pi tyA bnd')) = do
  -- verifies that the function's argument type is indeed a type
  tcType tyA
  -- unbinds the function's argument, body and body's return type
  (x, body, _, tyB) <- Unbound.unbind2Plus bnd bnd'
  -- checks that the function's body has the same type as the
  -- function's body's return type(provided by the Pi type)
  extendCtx (TypeSig x tyA) (checkType body tyB)
  return ty
tcTerm (Lam _) (Just nf) = err ["Lambda expression should have a function type, not", show nf]
-- ... rest of the terms ommited for brevity ...
-- if there's no specific case for *checking* the type of the term
tcTerm tm (Just ty) = do
  -- infer its type
  ty' <- inferType tm
  -- check if the inferred type is equal to the expected type
  ty `equate` ty'
  -- return the inferred type
  return ty'
\end{lstlisting}

\section{Examples}

The parsing code will be omitted for the sake of brevity.
We're now going to show some examples of the code in this programming language. First of all we're going to define the \code{Bool}, \code{Nat} and \code{Vec} data types along with \code{Nat}'s \code{plus} function:

\begin{piforall}
data Bool : Type where {
  False,
  True
}

data Nat : Type where {
  Zero,
  Succ of (Nat)
}

plus : (a b : Nat) -> Nat
plus = λa b. case a of {
  Zero -> b,
  Succ a' -> Succ (plus a' b)
}

data Vec (A : Type) (n : Nat) : Type where {
  Nil of [n = 0],
  Cons of (m : Nat) (A) (Vec A m) [n = Succ m]
}

empty_bool_vec : Vec Bool 0
empty_bool_vec = Nil

bool_vec : Vec Bool 2
bool_vec = Cons 1 False (Cons 0 True Nil)
\end{piforall}

Next, we're showing how to build a map function that works on vectors(note the type of the vectors having the parameterized length, on the map function type, showing dependent-types being used), we're buiding a vector and making a proof(static check) of its value:

\begin{piforall}
map : (A B : Type) -> (n : Nat) -> (f : (A -> B)) -> Vec A n -> Vec B n
map = λA B n f v. case v of {
  Nil -> Nil,
  Cons n' h t -> Cons n' (f h) (map A B n' f t)
}

nat_vec : Vec Nat 2
nat_vec = map Bool Nat 2 (λb. case b of {False -> 10, True -> 20}) bool_vec

-- proof that it is a vector like [10, 20]
p_nat_vec : nat_vec = (Cons 1 10 (Cons 0 20 Nil))
p_nat_vec = refl
\end{piforall}

Another use of dependent-types can be shown with the \code{concat} function, showing the returned vector having the length of the sum of the input vectors:

\begin{piforall}
concat : (A : Type) -> (m n: Nat) -> Vec A m -> Vec A n -> Vec A (plus m n)
concat = λA m n a b. case a of {
  Nil -> b,
  Cons m' h t -> Cons (plus m' n) h (concat A m' n t b)
}
\end{piforall}

The aforementioned \code{head} function can also be written in our language, along with a proof showing that the returned element is indeed the first element of the vector, and a comment stating a impossible term(the \code{head} of an empty vector)): 
\begin{piforall}
head : (A : Type) -> (n : Nat) -> Vec A (Succ n) -> A
head = λA n v. case v of {
  Cons _ h _ -> h
}

p_head : (head Nat 1 nat_vec) = 10
p_head = refl

-- the following lines do not type-check
-- because when we pass 0 as argument
-- the length of the expected vec argument is (Succ 0), which
-- does not match with the length of the actual vector(which is 0)
-- p_head_empty : (head Bool 0 empty_bool_vec) = False
-- p_head_empty = refl
\end{piforall}

\chapter{Conclusion}

In this work we presented a dependently-typed programming language, with a type-checker and a parser, written in Haskell. The language is based on the lambda calculus, and has a simple syntax, with a type system that supports dependent types. The language is not intended to be used in production, but rather as a proof-of-concept of a dependently-typed programming language, and as a way to learn more about dependently-typed programming languages. We believe that the language is simple enough to be used as a starting point for someone interested in learning about dependently-typed programming languages. The language can be found at \url{https://github.com/eduhenke/dep-tt}.


\postextual
\printbibliography{}

\end{document}